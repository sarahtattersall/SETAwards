\subsection{Creating a domain specific language}
In order to aid the programmatic creation of a Petri net, used primarily in unit-tests but also by users of our core libraries, we created a domain specific language (DSL). 
% A Domain Specific Language (DSL), is an easy to use language that has a 
% very specific purpose. For models it acts as a succinct, 
% readable mechanism for expressing how a model is configured and adds a
% layer of abstraction on top of the model~\cite{fowler2010domain}. 
The effectiveness of this DSL can be seen in \cref{lst:dsl} which reduces the number of lines required to construct a Petri net by a third and more importantly dramatically increases the readability.

\begin{figure}[tb]
\centering
\begin{subfigure}[b]{\linewidth}
\vspace{0.5cm}
\begin{lstlisting}[numbers=left, language=Java, frame=single]
PetriNet petriNet = new PetriNet();
Token defaultToken = new ColoredToken("Default", Color.BLACK);
Token redToken = new ColoredToken("Red", Color.RED);
Place p0 = new DiscretePlace("P0");
p0.setTokenCount("Default", 1);
Place p1 = new DiscretePlace("P1");
p1.setTokenCount("Red", 1);
p1.setTokenCount("Default", 2);
Transition t0 = new DiscreteTransition("T0");
t0.setTimed(false);
Transition t1 = new DiscreteTransition("T1");
t1.setTimed(true);
t1.setRate(new NormalRate("#(P0, Red)"));
Map<String, String> arc1Weights = new HashMap<>();
arc1Weights.put("Default", "4");
InboundArc arc1 = new InboundNormalArc(p0, t0, arc1Weights);
Map<String, String> arc2Weights = new HashMap<>();
arc2Weights.put("Default", "#(P1)*2");
OutboundArc arc2 = new OutboundNormalArc(t0, p0, arc2Weights);
petriNet.add(defaultToken);
petriNet.add(redToken);
petriNet.add(p0);
petriNet.add(p1);
petriNet.add(t0);
petriNet.add(t1);
petriNet.add(arc1);
petriNet.add(arc2);
\end{lstlisting}
\caption{Example code showing how to create Petri net components from their constructors in PIPE 5.}
\end{subfigure}

\begin{subfigure}[b]{\linewidth}
\vspace{0.5cm}
\begin{lstlisting}[numbers=left,language=Java,frame=single]
PetriNet petriNet = APetriNet.with(AToken.called("Default").withColor(Color.BLACK))
        .and(AToken.called("Red").withColor(Color.RED))
        .and(APlace.withId("P0").and(5, "Default").tokens())
        .and(APlace.withId("P1").and(1, "Red").and(2, "Default").to())
        .and(AnImmediateTransition.withId("T0"))
        .and(ATimedTransition.withId("T1").andRate("#(P0, Red)"))
        .and(ANormalArc.withSource("P0").andTarget("T0").and("4", "Default").tokens())
        .andFinally(ANormalArc.withSource("T0").andTarget("P0")
            .with("#(P1)*2", "Default").tokens());
\end{lstlisting}
\caption{Example code showing how to create a Petri net with the DSL provided with PIPE 5.}
\end{subfigure}

\caption{Comparison code emphasising that using the domain specific language (DSL) shipped in the PIPECore library not only reduces the lines taken to create a Petri net by a third, but also significantly increases the readability.}
\label{lst:dsl}
\end{figure}