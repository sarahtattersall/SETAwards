
\begin{table}[tb]
\begin{center}
  \begin{tabular}{| l | p{10cm} |}
    \hline
    Expression & Meaning \\ 
    \hline
    \texttt{\#(P0)} & The sum of all tokens in the place with id P0 \\
    
    % \hline
    % \texttt{\#(P0, Default)} & The number of Default tokens in the place with id P0 \\
    
    \hline
    \texttt{\#(P0, Red)*10} & The number of Red tokens in the place with id P0 multiplied by 10\\
    
    \hline
    \texttt{floor(10.5/3)} & the floor of $10.5/3$, i.e. $3$\\
    
    \hline
    \texttt{ceil(cap(P0) * 2.4)} & The ceiling of the capacity (max number of tokens allowed) in the place with id P0 multiplied by 2.5 \\

    % \hline
    % \texttt{\#(P0, Default)} & The number of Default tokens in the place with id P0 \\

    \hline
    \texttt{\#(P1) + \#(P2)} & The sum of the tokens in the places P1 and P2 \\
    \hline

  \end{tabular}
\caption{Example usage of PIPE 5 functional weights. It aims to resolve the ambiguity between places token counts by giving the user the choice of all tokens within a place or the count of a specific token within a place.}
\label{tbl:functional_weights}
\end{center}
\end{table}
