\section{A new software architecture for PIPE 5}
For PIPE 5 we performed a complete redesign of the back-end logic of a Petri net and released it as a stand alone core library and Maven artifact. 

We began by defining a set of interfaces that reflect a Petri net's internal components. In order to aid the design of the core Petri net structure we made use of several design patterns such as the acyclic visitor and builder patterns.
% \begin{sidewaysfigure}[tb]
\begin{center}
\setfigname{}\begin{tikzpicture} 
\umlclass[type=interface,x=0,y=0]{PetriNetComponent}{}{} 

\umlclass[type=interface,below=2cm of PetriNetComponent]{Token}{}{} 

\umlclass[type=interface,left=1cm of Token]{PlaceablePetriNetComponent}{}{}

\umlclass[type=interface, template={Source, Target}, right=1cm of Token]{Arc}{}{} 


\umlclass[type=interface,right=1cm of Arc]{RateParameter}{}{} 


\umlinherit[]{PlaceablePetriNetComponent}{PetriNetComponent}
\umlinherit[]{Token}{PetriNetComponent}
\umlinherit[]{Arc}{PetriNetComponent}
\umlinherit[]{RateParameter}{PetriNetComponent}

\umlclass[type=interface,below=2cm of PlaceablePetriNetComponent, xshift=-2cm]{Connectable}{}{} 
\umlinherit[]{Connectable}{PlaceablePetriNetComponent}


\umlclass[type=interface,below=2cm of Connectable, xshift=-1.5cm]{Place}{}{} 
\umlinherit[]{Place}{Connectable}

\umlclass[type=interface,right=1cm of Place]{Transition}{}{} 
\umlinherit[]{Transition}{Connectable}

\umlclass[type=interface,right=1cm of Connectable]{Annotation}{}{} 
\umlinherit[]{Annotation}{PlaceablePetriNetComponent}

\umlclass[type=interface,below=2cm of Arc, xshift=-1.6cm]{InboundArc}{}{}
\umlinherit[stereo={Place, Transition}, pos stereo=0.6]{InboundArc}{Arc}

\umlclass[type=interface,right=1cm of InboundArc]{OutboundArc}{}{}
\umlinherit[stereo={Transition, Place}, pos stereo=0.4]{OutboundArc}{Arc}
\end{tikzpicture}

\end{center}
\caption{UML diagram of the PIPE 5 Petri net components hierarchy. At the top of the hierarchy is the \texttt{PetriNetComponent} which contains methods common to every component. The \texttt{PlaceablePetriNetComponent} interface then defines a set of methods for storing the coordinates of components that can have (x, y) locations on the canvas. The \texttt{Connectable} interface describes a class which an arc can connect to and is subclassed by the \texttt{Place} and \texttt{Transition} interfaces which represent the nodes in the bipartite graph. Furthermore an \texttt{Arc} has been split into its connection type using Java generics and makes operations on different arc types much easier. PIPE 5 implements a set of concrete classes which extend these interfaces.}
\label{lst:petri_net_components_uml}
\end{sidewaysfigure}



In order to aid the programmatic creation of a Petri net, used primarily in unit-tests but also by users of our core libraries, we created a domain specific language (DSL). The clarity and effectiveness of this DSL can be seen in \cref{lst:dsl}.

\begin{figure}[tb]
\centering
\begin{subfigure}[b]{\linewidth}
\vspace{0.5cm}
\begin{lstlisting}[numbers=left, language=Java, frame=single]
PetriNet petriNet = new PetriNet();
Token defaultToken = new ColoredToken(``Default'', Color.BLACK);
Token redToken = new ColoredToken(``Red'', Color.RED);
Place p0 = new DiscretePlace(``P0'');
p0.setTokenCount(``Default'', 1);
Place p1 = new DiscretePlace(``P1'');
p1.setTokenCount(``Red'', 1);
p1.setTokenCount(``Default'', 2);
Transition t0 = new DiscreteTransition(``T0'');
t0.setTimed(false);
Transition t1 = new DiscreteTransition(``T1'');
t1.setTimed(true);
t1.setRate(new NormalRate(``#(P0, Red)''));
Map<String, String> arc1Weights = new HashMap<>();
arc1Weights.put(``Default'', `4'');
InboundArc arc1 = new InboundNormalArc(p0, t0, arc1Weights);
Map<String, String> arc2Weights = new HashMap<>();
arc2Weights.put(``Default'', ``#(P1)*2'');
OutboundArc arc2 = new OutboundNormalArc(t0, p0, arc2Weights);
petriNet.add(defaultToken); petriNet.add(redToken); petriNet.add(p0); petriNet.add(p1); petriNet.add(t0); petriNet.add(t1); petriNet.add(arc1); petriNet.add(arc2);
\end{lstlisting}
\caption{Example code showing how to create Petri net components from their constructors in PIPE 5.}
\end{subfigure}

\begin{subfigure}[b]{\linewidth}
\vspace{0.5cm}
\begin{lstlisting}[numbers=left,language=Java,frame=single]
PetriNet petriNet = APetriNet.with(AToken.called("Default").withColor(Color.BLACK))
        .and(AToken.called("Red").withColor(Color.RED))
        .and(APlace.withId("P0").and(5, "Default").tokens())
        .and(APlace.withId("P1").and(1, "Red").and(2, "Default").to())
        .and(AnImmediateTransition.withId("T0"))
        .and(ATimedTransition.withId("T1").andRate("#(P0, Red)"))
        .and(ANormalArc.withSource("P0").andTarget("T0").and("4", "Default").tokens())
        .andFinally(ANormalArc.withSource("T0").andTarget("P0")
            .with("#(P1)*2", "Default").tokens());
\end{lstlisting}
\caption{Example code showing how to create a Petri net with the DSL provided with PIPE 5.}
\end{subfigure}


\caption{Comparison code emphasising that using the domain specific language (DSL) shipped in the core library increases elegance and readability.}
\label{lst:dsl}
\end{figure}

Functional expressions allow an arcs associated weight, which is the number of tokens it will take or produce in a place, to alter based on properties of individual places in the Petri net. They are an important extension of Petri nets because they allow for more expressive models to be created.


The existing implementation of functional expressions was an area of concern due it being based on string manipulations, a poor API which was not tested and a lack of integration with the existing analysis modules meaning analysing Petri nets with functional expressions produced incorrect results or run-time errors. To overcome these problems we introduced a fully extensible context-free grammar using ANTLR v4 for Java, as shown in \cref{lst:bnf} and illustrated in \cref{tbl:functional_weights}.
\mediumlinespacing
\begin{figure}
\begin{lstlisting}[
    frame=single,
    language=Java]
grammar RateGrammar;

// PARSER
program : expression;

expression
    : '(' expression ')'                   # parenExpression
    | expression op=('*'|'/') expression   # multOrDiv
    | expression op=('+'|'-') expression   # addOrSubtract
    | 'ceil(' expression ')'               # ceil
    | 'floor(' expression ')'              # floor
    | capacity                             # placeCapacity
    | token_number                         # placeTokens
    | token_color_number                   # placeColorTokens
    | INT                                  # integer
    | DOUBLE                               # double;
capacity: 'cap(' ID ')';
token_number: '#(' ID ')';
token_color_number: '#(' ID ',' ID ')';

// LEXER
ID     : ('a'..'z'|'A'..'Z') ('a'..'z'|'A'..'Z'|'0'..'9')*;
INT    : '0'..'9'+;
DOUBLE  : '0'..'9'+ '.' '0'..'9'+;
WS     : [ \t\n\r]+ -> skip ;
    
\end{lstlisting}
\caption{PIPE 5's new BNF grammar for functional expressions, written for ANTLR v4.}

\label{lst:bnf},
\end{figure}
\normallinespacing

\begin{table}[tb]
\begin{center}
  \begin{tabular}{| l | p{10cm} |}
    \hline
    Expression & Meaning \\ 
    \hline
    \texttt{\#(P0)} & The sum of all tokens in the place with id P0 \\
    
    % \hline
    % \texttt{\#(P0, Default)} & The number of Default tokens in the place with id P0 \\
    
    \hline
    \texttt{\#(P0, Red)*10} & The number of Red tokens in the place with id P0 multiplied by 10\\
    
    \hline
    \texttt{floor(10.5/3)} & the floor of $10.5/3$, i.e. $3$\\
    
    \hline
    \texttt{ceil(cap(P0) * 2.4)} & The ceiling of the capacity (max number of tokens allowed) in the place with id P0 multiplied by 2.5 \\

    % \hline
    % \texttt{\#(P0, Default)} & The number of Default tokens in the place with id P0 \\

    \hline
    \texttt{\#(P1) + \#(P2)} & The sum of the tokens in the places P1 and P2 \\
    \hline

  \end{tabular}
\caption{Example usage of PIPE 5 functional weights.}
\label{tbl:functional_weights}
\end{center}
\end{table}

