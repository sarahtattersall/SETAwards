\section{What is a Petri net?}
A Stochastic Petri net,  as seen in \cref{tikz:photo}, is a bipartite graph made up of places, timed transitions, arcs and tokens. The main strengths of Petri nets are their convenient graphical representation and their natural expressions of synchronisation. They are often chosen over Markov chains to model real world problems because of their strengths and the isomorphism that exists between them. They can be mathematically analysed to yield useful information about a system such as the percentage of time spent in each state. Interesting applications include the modelling of computer virus life cycles~\cite{kotenko2005analyzing}, modelling information flow in security policies to determine the possible ways in which information may be compromised~\cite{varadharajan1990petri} and modelling of extreme unforeseen events, such as very heavy rainfall at water sewage plants which can lead to waste water overflowing into the streets~\cite{ghasemieh2013analysis}.

\begin{figure}[tb]
\setfigname{}
\begin{tikzpicture}[node distance=2.5cm,>=stealth',bend angle=45,auto]
    \node [place,tokens=1, label=south:dark] (dark) {};

    \node [timed transition,label=north:sunrise] (sunrise) [above right of=dark] {}
        edge [pre]  (dark);

    \node [timed transition,label=north:sunset] (sunset) [below right of=dark] {}
        edge [pre]  (dark);

    \node [place,tokens=1, label=south:H2O] (h2o) [above left of = sunrise] {};
        % edge [post] (t3);

    \node [timed transition,label=north:photosynthesis, xshift=1.75cm] (photo) [above right of=h2o] {}
    edge [pre]  (h2o);    


    \node [place,tokens=1, label=south:CO2, xshift=-1.75cm] (co2) [above left of = photo] {}
    edge [post] (photo);

    \node [place, label=south:light] (light) [above right of = sunset] {}
    edge [pre] (sunset)
    edge [pre] (sunrise)
    edge [pre] (photo)
    edge [post] (photo);

    \node [place, label=south:C6H12O6, xshift=1.75cm] (sugar) [above right of = photo] {}
    edge [pre] (photo);

    \node [place, label=south:O2,xshift=1.75cm] (o2) [below right of = photo] {}
    edge [pre] (photo);
        % edge [post] (t3);


    \node [timed transition,label=north:metabolism, xshift=1.75cm] (met) [above right of=o2] {}
    edge [pre] (sugar)
    edge [pre]  (o2);   

    \node[coordinate, yshift=0.2cm] (empty1) [right=0.8cm of met]{};
    \node[coordinate, yshift=-0.2cm] (empty2) [right=0.8cm of met]{};
    \node[coordinate] (empty3) [above=0.5cm of sugar]{};
    \node[coordinate] (empty3) [above=0.5cm of co2]{};
    \node[coordinate] (empty4) [below=0.5cm of sunset]{};
    \node[coordinate] (empty5) [left=0.5cm of co2]{};
    \node[coordinate] (empty6) [left=0.5cm of h2o]{};



  \draw[->]
      (met) -- (empty1) |- (empty3) -| (empty5) -- (co2);
  \draw[->]
      (met) -- (empty2) |- (empty4) -|  (empty6) -- (h2o);



      % \node [place,label=west:$p_2$] (p2) [below right of = t1] {}
      %   edge [pre]  (t1);

      % \node [timed transition,label=east:$t_2$] (t2) [below left of=p2] {}
      %   edge [pre]  (p2);


      % \node [timed transition, label=east:$t_3$] (t3) [below of=t2] {};
      
      % \node [place,tokens=1, label=west:$p_3$] (p3) [below left of = t3] {}
      %   edge [post] (t3);

      % \node [place, label=west:$p_4$] (p4) [below right of = t3] {}
      %   edge [pre] (t3);

      % \node [timed transition,label=east:$t_4$] (t4) [below left of=p4] {}
      %   edge [pre]  (p4);

      % \draw[
      %   >=latex,
      %   auto=right,                      % or
      %   loop above/.style={out=75,in=105,loop},
      %   every loop,
      %   ]
      %    (t2)   edge node {}   (p1)
      %    (t4)   edge node {}   (p3);
\end{tikzpicture} 
\caption{A Petri net representing the photosynthesis chemical reaction. In the Petri net the blue circles are known as places and represent the resources that can be consumed and created in the reaction. The smaller black circles in the places represent the current resources available in the system. The rectangles represent actions that can happen in the chemical reaction, for example sunrise. Arcs connect places to transitions and transitions to places in order to make up a bipartite graph. During animation when an action takes place a token is removed from any inbound places and tokens are produced in the outbound places. In order for an action to take place there must be a token in every inbound place, for example the current state photosynthesis cannot take place because it is still dark outside, it requires the sun to rise in order to happen. }
\label{tikz:photo}
\end{figure}