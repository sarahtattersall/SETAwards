\begin{table}[tb]
\begin{center}
  \begin{tabular}{| c | c | c | c | c | }
  \hline
    Number of states & Number of transitions & PIPE 4 (s) & PIPE 5 (s) & Speedup \\
    \hline
    40 & 156 & 0.21 & 0.17 & 1.24\\
    \hline
    100 & 480 & 0.36 & 0.40 & 0.90\\
    \hline
    \num{625} & \num{4000} & \num{25.12} & \num{1.35} & \num{18.61}\\
    \hline
    \num{1350} & \num{9450} & \num{83.67} & \num{1.75} & \num{47.81}\\
    \hline
    \num{4096} & \num{28672} & \num{728.02} & \num{3.82} & \num{190.58}\\
    \hline
    \num{11664} & \num{93312} & \num{2738.37} & \num{8.51} & \num{321.78}\\
    \hline
  \end{tabular}
\caption{The time taken in seconds to generate the reachability graph in PIPE 4 compared with the new sequential algorithm in PIPE 5. For very small state spaces we can see that the times are comparable, but for moderate sized Petri nets PIPE 5 is more scalable.}
\label{tbl:pipe5_vs_pipe4_sequential}
\end{center}
\end{table}