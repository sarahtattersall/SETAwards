\subsection{Static analysis}
By analysing the separate repositories we can see that the previous count of \num{12904} reported issues for PIPE 4 has been considerably reduced to a total of \num{937} issues. The break down of quality issues across repositories can be seen in \cref{tbl:pipe5_qaplug}.

\begin{table}[tb]
\small
\begin{center}
  \begin{tabular}{| l | c | c | c | c | c | c |}
    \hline
    Repository & Efficiency & Maintainability & Portability & Reliability & Usability & \textbf{Total} \\ 
    \hline
    PIPEMarkovChain & 0 & 0 & 0 & 0 & 0 & \textbf{0}\\ 
    \hline
    PIPEAnalysis & 0 & 0 & 0 & 0 & 0 & \textbf{0}\\    
    \hline
    PIPECore & 0 & 19 & 0 & 68 & 294 & \textbf{381}\\
    \hline
    PIPE & 13 & 77 & 0 & 426 & 40 & \textbf{556}\\
    \hline
    \textbf{Total} & \textbf{13} & \textbf{96} & \textbf{0} & \textbf{494} & \textbf{334} & {\color{red}\textbf{937}}\\
    \hline

  \end{tabular}
\caption{Break down of issues highlighted by the QAPlug analysis plug-in for Intellij for
each of the new PIPE 5 repositories. The total number of code quality issues for the entire project has been reduced from \num{12904} to \num{937}. Of these \num{937} issues \num{349} are due to auto-generated code via the ANTLR v4 plug-in.
On analysing these code quality issues we found that \num{349} of them are due to the ANTLR v4 auto-generated code meaning that the number of code quality issues for our written code is actually \num{588}. 
Of the remaining issues in the view codebase a total of \num{473} come from the \textit{`Magic Number Count'} metric. These are entirely due to layout and sizing settings for view components and indicate that whilst a useful metric for non-view code it is perhaps not so relevant for projects containing GUI code.
}
\label{tbl:pipe5_qaplug}
\end{center}
\end{table}

Furthermore Stan4J reported that the project size has reduced by a third from around \num{66000} lines of code to \num{22000} which is far more maintainable.


