\begin{figure}[H]
\small{
\begin{center}
\setfigname{}
\internal{}
\begin{tikzpicture}
  \begin{axis}[axis equal image, enlargelimits=false, axis lines=none,
    clip=false, scale=1.2]

  % offset, brace amplitude and brace style
  \def\offset{5}
  \def\amp{10pt}
  \def\xmax{1227}
  \def\ymax{720}
  \tikzset{brace/.style={draw, thick, decorate, decoration={brace, amplitude=\amp}}}
  \tikzset{brace_mirror/.style={brace, decoration={mirror}}}
  \tikzset{brace_node/.style={midway, text centered}}
  \tikzset{label/.style={draw, stealth-, thick}}
  \tikzset{arrow_node/.style={text centered}}

  \addplot graphics [xmin=0,xmax=\xmax,ymin=0,ymax=\ymax] {analysis/edit_mode.png};

  %%%%%%%%%%% BRACES %%%%%%%%%%%
  % Modules
  \draw[brace_mirror] (axis cs:-\offset,624) -- (axis cs:-\offset,338) node[brace_node, left, text width=1.5cm, xshift=-\amp] {Plug-in modules};

  % Components
  \draw[brace] (axis cs:650,676)--(axis cs:826,676) node[brace_node, above, text width=2.4cm, yshift=\amp] {Petri net component creators};

  % Token
  \draw[brace] (axis cs:840,676)--(axis cs:1050,676) node[brace_node, above, text width=3cm, yshift=\amp] {Token editing};

  % Component manipulation
  \draw[brace] (axis cs:193,676)--(axis cs:362,676) node[brace_node, above, text width=2.5cm, yshift=\amp] {Component editing};

  % Zoom manipulation
  \draw[brace] (axis cs:371,676)--(axis cs:496,676) node[brace_node, above, text width=1.5cm, yshift=\amp] {Zoom features};

  % File editing 
  \draw[brace] (axis cs:5,676)--(axis cs:142,676) node[brace_node, above, text width=2.5cm, yshift=\amp] {File editing};

  %%%%%%%%%%% Arrows %%%%%%%%%%%%%
  % Animate button
  \draw[label] (axis cs:576,666)|-++(-4mm,17mm) node[arrow_node, left] {Toggle animate mode};
  
  % Select button
  \draw[label] (axis cs:620,666)--++(0,20mm) node[arrow_node, above, text width=3cm] {Select components mode};

  % Add rate parameter button
  \draw[label] (axis cs:1160,666)--++(0,20mm) node[arrow_node, above, text width=3cm] {Add rate parameter};

  \end{axis}
\end{tikzpicture}
\end{center}
}
\caption{A labelled diagram of PIPE 4 in editing mode.}
\label{fig:edit}
\end{figure}