% Stochastic Petri nets are a high-level formalism for modelling and analysing the behaviour of complex concurrent systems.
% The Platform Independent Petri net Editor (PIPE) is an open-source tool for building and evaluating them. Since 2003 features have been added to PIPE by many contributors with a general lack of consideration for the quality of the software engineering and the impact on the correctness of other features. This project presents a complete renewal of PIPE's underlying software architecture as well as new analysis modules which exploit the features of multi-core processors to achieve over 300x speedups. 
This project concerns the analysis, reengineering and extension of the
substantial codebase underlying the Platform Independent Petri net
Editor (PIPE), a popular package for the modelling of concurrent
systems. PIPE's codebase had evolved in a largely uncontrolled fashion
since its creation in 2002, resulting in a tangled, bloated and
bug-ridden codebase. This project has renewed the software
architecture, added rigorous support for functional expressions, added
an extensive test suite, streamlined and modernised the build process
and added quantitative analysis modules suitable for multicore
processors.  With 66\% fewer lines of code, the new codebase is
significantly leaner and easier to maintain.