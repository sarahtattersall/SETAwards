\section{What is PIPE?}
PIPE, shown in \cref{fig:edit}, is a graphical tool for creating, simulating and analysing Petri nets. In October 2013 the latest release was known as PIPE 4; this version currently has over \num{90000} downloads to date on SourceForge~\cite{downloads}. 

% Whilst the graphical user interface (GUI), seen in \cref{fig:edit}, is easy to use the underlying software architecture has become bloated and riddled with bugs.

\begin{figure}[H]
\setfigname{}
\begin{tikzpicture}[node distance=2.5cm,>=stealth',bend angle=45,auto]
    \node [place,tokens=1, label=south:dark] (dark) {};

    \node [timed transition,label=north:sunrise] (sunrise) [above right of=dark] {}
        edge [pre]  (dark);

    \node [timed transition,label=north:sunset] (sunset) [below right of=dark] {}
        edge [pre]  (dark);

    \node [place,tokens=1, label=south:H2O] (h2o) [above left of = sunrise] {};
        % edge [post] (t3);

    \node [timed transition,label=north:photosynthesis, xshift=1.75cm] (photo) [above right of=h2o] {}
    edge [pre]  (h2o);    


    \node [place,tokens=1, label=south:CO2, xshift=-1.75cm] (co2) [above left of = photo] {}
    edge [post] (photo);

    \node [place, label=south:light] (light) [above right of = sunset] {}
    edge [pre] (sunset)
    edge [pre] (sunrise)
    edge [pre] (photo)
    edge [post] (photo);

    \node [place, label=south:C6H12O6, xshift=1.75cm] (sugar) [above right of = photo] {}
    edge [pre] (photo);

    \node [place, label=south:O2,xshift=1.75cm] (o2) [below right of = photo] {}
    edge [pre] (photo);
        % edge [post] (t3);


    \node [timed transition,label=north:metabolism, xshift=1.75cm] (met) [above right of=o2] {}
    edge [pre] (sugar)
    edge [pre]  (o2);   

    \node[coordinate, yshift=0.2cm] (empty1) [right=0.8cm of met]{};
    \node[coordinate, yshift=-0.2cm] (empty2) [right=0.8cm of met]{};
    \node[coordinate] (empty3) [above=0.5cm of sugar]{};
    \node[coordinate] (empty3) [above=0.5cm of co2]{};
    \node[coordinate] (empty4) [below=0.5cm of sunset]{};
    \node[coordinate] (empty5) [left=0.5cm of co2]{};
    \node[coordinate] (empty6) [left=0.5cm of h2o]{};



  \draw[->]
      (met) -- (empty1) |- (empty3) -| (empty5) -- (co2);
  \draw[->]
      (met) -- (empty2) |- (empty4) -|  (empty6) -- (h2o);



      % \node [place,label=west:$p_2$] (p2) [below right of = t1] {}
      %   edge [pre]  (t1);

      % \node [timed transition,label=east:$t_2$] (t2) [below left of=p2] {}
      %   edge [pre]  (p2);


      % \node [timed transition, label=east:$t_3$] (t3) [below of=t2] {};
      
      % \node [place,tokens=1, label=west:$p_3$] (p3) [below left of = t3] {}
      %   edge [post] (t3);

      % \node [place, label=west:$p_4$] (p4) [below right of = t3] {}
      %   edge [pre] (t3);

      % \node [timed transition,label=east:$t_4$] (t4) [below left of=p4] {}
      %   edge [pre]  (p4);

      % \draw[
      %   >=latex,
      %   auto=right,                      % or
      %   loop above/.style={out=75,in=105,loop},
      %   every loop,
      %   ]
      %    (t2)   edge node {}   (p1)
      %    (t4)   edge node {}   (p3);
\end{tikzpicture} 
\caption{A Petri net representing the photosynthesis process. The blue circles are known as places and are resource holders, the smaller black circles are known as tokens and represent resources. Transitions are the rectangles and represent actions. Arcs connect places to transitions and transitions to places.}
% During animation when an action takes place the specified number of tokens are removed from any inbound places and tokens are produced in the outbound places, if there is no weight on an arc the default value is 1. In order for an action to take place there must be a token in every inbound place, for example the current state photosynthesis cannot take place because it is still dark outside, it requires the sun to rise in order to happen. }
\label{tikz:photo}
\end{figure}

\begin{figure}[H]
\small{
\begin{center}
\setfigname{}
\internal{}
\begin{tikzpicture}
  \begin{axis}[axis equal image, enlargelimits=false, axis lines=none,
    clip=false, scale=1.2]

  % offset, brace amplitude and brace style
  \def\offset{5}
  \def\amp{10pt}
  \def\xmax{1227}
  \def\ymax{720}
  \tikzset{brace/.style={draw, thick, decorate, decoration={brace, amplitude=\amp}}}
  \tikzset{brace_mirror/.style={brace, decoration={mirror}}}
  \tikzset{brace_node/.style={midway, text centered}}
  \tikzset{label/.style={draw, stealth-, thick}}
  \tikzset{arrow_node/.style={text centered}}

  \addplot graphics [xmin=0,xmax=\xmax,ymin=0,ymax=\ymax] {analysis/edit_mode.png};

  %%%%%%%%%%% BRACES %%%%%%%%%%%
  % Modules
  \draw[brace_mirror] (axis cs:-\offset,624) -- (axis cs:-\offset,338) node[brace_node, left, text width=1.5cm, xshift=-\amp] {Plug-in modules};

  % Components
  \draw[brace] (axis cs:650,676)--(axis cs:826,676) node[brace_node, above, text width=2.4cm, yshift=\amp] {Petri net component creators};

  % Token
  \draw[brace] (axis cs:840,676)--(axis cs:1050,676) node[brace_node, above, text width=3cm, yshift=\amp] {Token editing};

  % Component manipulation
  \draw[brace] (axis cs:193,676)--(axis cs:362,676) node[brace_node, above, text width=2.5cm, yshift=\amp] {Component editing};

  % Zoom manipulation
  \draw[brace] (axis cs:371,676)--(axis cs:496,676) node[brace_node, above, text width=1.5cm, yshift=\amp] {Zoom features};

  % File editing 
  \draw[brace] (axis cs:5,676)--(axis cs:142,676) node[brace_node, above, text width=2.5cm, yshift=\amp] {File editing};

  %%%%%%%%%%% Arrows %%%%%%%%%%%%%
  % Animate button
  \draw[label] (axis cs:576,666)|-++(-4mm,17mm) node[arrow_node, left] {Toggle animate mode};
  
  % Select button
  \draw[label] (axis cs:620,666)--++(0,20mm) node[arrow_node, above, text width=3cm] {Select components mode};

  % Add rate parameter button
  \draw[label] (axis cs:1160,666)--++(0,20mm) node[arrow_node, above, text width=3cm] {Add rate parameter};

  \end{axis}
\end{tikzpicture}
\end{center}
}
\caption{A labelled diagram of PIPE 4 in editing mode.}
\label{fig:edit}
\end{figure}